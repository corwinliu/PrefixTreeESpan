% !Mode:: "TeX:UTF-8"
% +-----------------------------------------------------------------------------
% | File: report
% | Author: huxuan
% | E-mail: i(at)huxuan.org
% | Created: 2012-11-13
% | Last modified: 2012-11-13
% | Description:
% |     Description for report
% |
% | Copyrgiht (c) 2012 by huxuan. All rights reserved.
% | License GPLv3
% +-----------------------------------------------------------------------------

\documentclass{ctexart}
\usepackage[top=1in,bottom=1in,left=1.25in,right=1.25in]{geometry}

\renewcommand{\baselinestretch}{1.5}

\let\stdsection\section
\renewcommand\section{\newpage\stdsection}

\usepackage{color}
\definecolor{dkgreen}{rgb}{0,0.6,0}
\definecolor{gray}{rgb}{0.5,0.5,0.5}
\definecolor{mauve}{rgb}{0.58,0,0.82}

\usepackage[ruled,vlined]{algorithm2e}

\usepackage{hyperref}
\hypersetup{colorlinks}

\usepackage{listings}
\renewcommand{\lstlistingname}{代码}
\lstset{
    backgroundcolor=\color{white},
    basicstyle=\zihao{5}\ttfamily,
    columns=flexible,
    breakatwhitespace=false,
    breaklines=true,
    captionpos=b,
    frame=single,
    numbers=left,
    numbersep=5pt,
    showspaces=false,
    showstringspaces=false,
    showtabs=false,
    stepnumber=1,
    rulecolor=\color{black},
    tabsize=2,
    % texcl=true,
    title=\lstname,
    escapeinside={\%*}{*)},
    extendedchars=false,
    mathescape=true,
    xleftmargin=3em,
    xrightmargin=3em,
    numberstyle=\color{gray},
    keywordstyle=\color{blue},
    commentstyle=\color{dkgreen},
    stringstyle=\color{mauve},
}

\RequirePackage{dirtree}
\renewcommand*\DTstylecomment{\ttfamily\color{dkgreen}}
\renewcommand*\DTstyle{\ttfamily\color{red}}

\title{PrefixTreeESpan算法实验报告}
\author{扈煊 1201214133}
\date{\today}

\begin{document}

\maketitle

\section{实验目的}

\begin{itemize}
    \item 理解并实现基于模式增长的频繁子树挖掘算法 PrefixTreeESpan
        \cite{zou2006prefixtreeespan}
    \item[] ({\bf Prefix}-{\bf Tree}-projected
        {\bf E}mbedded-{\bf S}ubtree {\bf pa}tter{\bf n})
    \item 验证算法基本正确性并进行可能的优化和改进
\end{itemize}

\section{作业说明}

\subsection{目录文件结构}

本次作业打包包含的内容及其说明如下:

{
    \dirtree{%
        .1 PrefixTreeESpan.
        .2 main.py \DTcomment{算法实现主程序}.
        .2 tree.py \DTcomment{辅助类tree.Node}.
        .2 project.py \DTcomment{辅助类project.Project}.
        .2 README.txt \DTcomment{简要说明}.
        .2 test.in \DTcomment{demo 测试文件}.
        .2 Makefile \DTcomment{Makefile 脚本,用于make方式运行命令}.
        .2 data/ \DTcomment{数据集目录}.
        .3 CSlog.data.
        .3 D10.data.
        .3 F5.data.
        .3 T1M.data.
        .2 output/ \DTcomment{输出结果目录,对应不同的测试编号}.
        .3 CSlog-1.out.
        .3 D10-1.out.
        .3 D10-2.out.
        .3 D10-3.out.
        .3 F5-1.out.
        .3 F5-2.out.
        .3 F5-3.out.
        .2 report/ \DTcomment{实验报告目录}.
        .3 report.tex \DTcomment{\LaTeX{}源文件}.
        .3 report.bib \DTcomment{BiB\TeX{}参考文献文件}.
        .3 Makefile \DTcomment{\TeX{}make脚本}.
    }
}

\subsection{运行方法}

\subsubsection{Makefile}

Makefile 脚本的运行方法正对特定的数据集,需要支持make命令的环境,
即\ref{subsubsec-result}节实验结果中对应的不同测试样例,
运行方法为"make 测试编号",如测试编号为“CSlog-1”,则运行命令为“make CSlog-1”,
具体测试编号请见表\ref{tab-result}。

\subsubsection{Python}

Python 脚本的运行方法针对通用的数据集,可以自定义输入文件、输出文件和最小支持度

\begin{description}
    \item [命令行] python main.py -i inputfile -o outputfile -s minsup
    \item [参数说明] ~
    \begin{description}
        \item[-i] 必选,输入文件,即图数据库对应文件,
            默认值为"test.in",为论文中提到的实例
        \item[-o] 可选,输出文件,即结果输出文件,默认值为"test.out"
        \item[-s] 必选,最小支持度,整型,默认值为"2"
    \end{description}
\end{description}

\section{算法设计}

\subsection{主要思想}

\begin{itemize}
    \item 频繁子树的推导子树一定是频繁的
    \item 频繁子树总是可以通过其前缀树增长得到
    \item 利用深度优先搜索的思想通过递归迭代不断对前缀树进行增长,
        统计频繁的增长因子,从而得到更多的频繁子树
\end{itemize}

\subsection{伪代码}

\begin{algorithm}[H]
    \caption{PrefixTreeESpan()}
    \KwData{树数据库 $TreeDB$, 最小支持计数 $MINSUP$}

    利用栈深度优先遍历每棵树中所有节点,计算配对范围 \;
    利用集合合并每棵树中的标签,统计标签计数 \;
    将每个标签计数与最小支持计数比较得一阶频繁标签集合 \;
    \For{一阶频繁子树 $in$ 一阶频繁子树集合}{
        保存和输出一阶频繁子树即前缀树 $pre\_tree$ \;
        \For{一阶频繁子树在数据库 $TreeDB$ 中的出现}{
            对于每一例出现生成其对应的投影实例 \;
            将投影实例加入到新的投影数据库中 $ProDB$ \;
        }
        调用函数$get\_fre$ ( $pre\_tree$, $1$, $ProDB$ ) \;
    }
\end{algorithm}

\begin{algorithm}[H]
    \caption{$get\_fre$ ( $pre\_tree$, $n$, $ProDB$ )}

    利用集合统计所有投影实例中的增长因子计数 \;
    将每个增长因子计数与最小支持计数比较得到频繁增长因子 \;
    \For{$n+1$阶增长因子 $in$ $n+1$阶增长因子集合}{
        生成、保存并输出$n+1$阶频繁子树即前缀树 $pre\_tree\_new$ \;
        \For{$n+1$阶增长因子在投影数据库$ProDB$中的出现}{
            对于每一例出现生成其对应的投影实例 \;
            将投影实例加入到心的投影数据库中 $ProDB\_new$ \;
        }
        调用函数$get\_fre$ ( $pre\_tree\_new$, $n+1$, $ProDB\_new$ ) \;
    }
\end{algorithm}

% \begin{algorithm}[H]
%     \caption{PrefixTreeESpan()}
%     \begin{algorithmic}[1]
%         \REQUIRE 树数据库 $TreeDB$, 最小支持计数 $MINSUP$
%         \STATE 利用栈深度优先遍历每棵树中所有节点,计算配对范围
%         \STATE 利用集合合并每棵树中的标签,统计标签计数
%         \STATE 将每个标签计数与最小支持计数比较得一阶频繁标签集合
%         \FORALL{一阶频繁子树 $in$ 一阶频繁子树集合}
%             \STATE 保存和输出一阶频繁子树即前缀树 $pre\_tree$
%             \FORALL{一阶频繁子树在数据库 $TreeDB$ 中的出现}
%                 \STATE 对于每一例出现生成其对应的投影实例
%                 \STATE 将投影实例加入到新的投影数据库中 $ProDB$
%             \ENDFOR
%             \STATE 调用函数$get\_fre$ ( $pre\_tree$, $1$, $ProDB$ )
%         \ENDFOR
%     \end{algorithmic}
% \end{algorithm}
% 
% \begin{algorithm}[H]
%     \caption{$get\_fre$ ( $pre\_tree$, $n$, $ProDB$ )}
%     \begin{algorithmic}[1]
%         \STATE 利用集合统计所有投影实例中的增长因子计数
%         \STATE 将每个增长因子计数与最小支持计数比较得到频繁增长因子
%         \FORALL{$n+1$阶增长因子 $in$ $n+1$阶增长因子集合}
%             \STATE 生成、保存并输出$n+1$阶频繁子树即前缀树 $pre\_tree\_new$
%             \FORALL{$n+1$阶增长因子在投影数据库$ProDB$中的出现}
%                 \STATE 对于每一例出现生成其对应的投影实例
%                 \STATE 将投影实例加入到心的投影数据库中 $ProDB\_new$
%             \ENDFOR
%             \STATE 调用函数$get\_fre$ ( $pre\_tree\_new$, $n+1$, $ProDB\_new$ )
%         \ENDFOR
%     \end{algorithmic}
% \end{algorithm}

\section{算法实现}

\subsection{算法实现概述}

本次实验使用了Python语言进行了代码的编写,
采用Python语言的主要原因除了个人使用比较熟练以外,
还考虑到可以使用Python中的高级数据结构,如列表、字典和集合,来简化一些处理。
并且在处理命令行参数和字符串处理等方面,Python包装的很好使用起来更顺手一些。

\subsection{辅助类}

\begin{description}
    \item[tree.Node] 表示树中的每一个节点,记录其标号和匹配范围
    \begin{description}
        \item[Node.label] 表示该节点的标号
        \item[Node.label] 表示该节点对应的匹配位置范围,即与其对应的"-1"的序号
    \end{description}
    \item[project.Project] 封装为一个投影数据库的信息,
        主要参考了论文中所述的伪投影法,为了方便使用有所修改。
    \begin{description}
        \item[Project.tid] 表示该投影数据库所在的树的序号
        \item[Project.offset\_list] 表示该投影数据库每一部分在树中的偏移量列表
        \item[Project.scope\_list] 表示该投影数据库每一部分在树中每一个偏移量
            对应父节点的匹配位置范围列表
    \end{description}
\end{description}

\subsection{伪投影数据库}

论文原文中的伪投影数据库采用的是三元组的形式,但是不难发现tid,
即树的编号是重复的,因为投影数据库只可能是某一棵树的子树或森林,
而不可能跨越不同的树,所以本次实现中将投影数据库封装成了一个类,
其中tid表示该树的序号,三元组中的其他两个元素分别用列表对应存储。
具体实现见代码\ref{code-project}。

\section{实验结果}

\subsection{实验环境}

\begin{description}
    \item[CPU] Intel\textsuperscript{\textregistered}
        Core\textsuperscript{\texttrademark}2 Duo CPU T7500 @ 2.20GHz × 2
    \item[RAM] 1G * 2
    \item[OS]  Ubuntu-Desktop 12.10 64-bit
\end{description}

\subsection{测试数据}
\label{subsec-test-data}

本次测试选用了老师提供的4个测试集,
每个测试集的文件大小及其包含树的个数见表\ref{tab-test-data}

\begin{table}[!h]
    \centering
    \caption{数据集文件大小及其包含树的个数}
    \label{tab-test-data}
    \begin{tabular}{c|c|c}
        \hline
        数据集  & 文件大小(M) & 树的个数 \\ \hline
        CSlog   &  6.0        &   59691 \\ \hline
        D10     &  1.7        &  100000 \\ \hline
        F5      &  1.8        &  100000 \\ \hline
        T1M     & 15.0        & 1000000 \\ \hline
    \end{tabular}
\end{table}

\subsection{实验结果}
\label{subsec-result}

本次实验主要针对\ref{subsec-test-data}节中提到的数据集进行了测试,
每个数据集分别选用了最小支持度百分比为10\%、1\%和0.1\%三个比例进行了测试,
所有输出结果都在output文件夹中,对应的文件名为“测试编号.out”,
如测试编号D10-1对应的输出文件为“D10-1.out”,输出内容为所有挖掘出来的频繁子树,
最后一行为所消耗的时间,得到的统计结果见表\ref{tab-result}。

\begin{table}[h!]
    \centering
    \caption{实验结果}
    \label{tab-result}
    \begin{tabular}[h!]{l|l|r|r|r|c}
        \hline
        数据集 & 测试编号 & MINSUP 个数 & MINSUP 百分比 & 频繁子树个数 & 运行时间 \\ \hline
        CSlog & CSlog-1 & 5969 & 10 & 2 & 00:06.95 \\ \hline
        D10 & D10-1 & 10000 & 10 & 6 & 00:04.17 \\ \hline
        D10 & D10-2 & 1000 & 1 & 405 & 00:22.13 \\ \hline
        D10 & D10-3 & 100 & 0.1 & 7426 & 00:59.79 \\ \hline
        F5 & F5-1 & 10000 & 10 & 16 & 00:06.11 \\ \hline
        F5 & F5-2 & 1000 & 1 & 180 & 00:15.30 \\ \hline
        F5 & F5-3 & 100 & 0.1 & 928 & 00:25.72 \\ \hline
    \end{tabular}
\end{table}

\section{实验总结}

\subsection{关于 T1M}

比较实验数据和实验结果,不难看出没有T1M数据集相关的结论,
在试验过程中运行T1M数据集时,内存占用会迅速膨胀,
不用一会就会将整个计算机的内存耗尽,所以就没有让程序继续运行。
CSlog数据集的最小支持度为1\%和0.1\%的测试用例也是同样的原因没有得到最终结果。
主要原因应该有以下几个方面:
\begin{itemize}
    \item Python作为解释型脚本语言,本身运行时就会占用较多内存,
        对内存的要求也比较高
    \item 算法中使用了递归的逻辑,没有进行非递归实现
    \item 算法实现过程中使用了较多高级数据结构,尤其在递归中还作为了参数,
        用来当前的前缀树和投影数据库
\end{itemize}

\subsection{关于效率}

出本次算法实现的效率和论文中已经实现的方法相比总体偏低,
个人觉得主要原因有如下几点:
\begin{itemize}
    \item Python 语言本身执行效率没有C/C++之类的语言高
    \item 未实现论文中已提到的另一个优化,提前减枝,删去数据中不频繁的节点
    \item 算法实现的其他细节还没有充分优化,比如文件输出、内存清空等
\end{itemize}

\section{源代码及注释}

\lstinputlisting[
    language={Python},
    label={code-tree},
    caption={辅助类tree.Node},
]{../tree.py}

\lstinputlisting[
    language={Python},
    label={code-project},
    caption={辅助类project.Project},
]{../project.py}

\lstinputlisting[
    language={Python},
    label={code-prefixtreeespan},
    caption={PrefixTreeESpan算法实现主体},
]{../main.py}

\bibliographystyle{plain}
\bibliography{report}
\addcontentsline{toc}{chapter}{参考文献}

\end{document}
